\documentclass{article}
\usepackage[utf8]{inputenc}
\usepackage{algorithm}
\usepackage{algorithmic}

\title{Report}
\author{Patrik-Tibor Csanyi/ptc1g20/31574602}
\date{May 2021}

\begin{document}

\maketitle

\section{Introduction}
In this report we attempt to take a csv file, introduce the data from it into a database, normalise the previously mentioned database, and then perform queries on it.
\section{The Relational Model}
\subsection{EX1}
\begin{tabular}{|c|c|}
     \hline
     \multicolumn{2}{|c|}{Dataset} \\
     \hline dateRep & NUMERIC \\
     \hline day & INTEGER \\
     \hline month & INTEGER \\
     \hline year & INTEGER \\
     \hline cases & INTEGER \\
     \hline deaths & INTEGER \\
     \hline countriesAndTerritories & TEXT \\
     \hline geoId & TEXT \\
     \hline countryterritoryCode & TEXT \\
     \hline popData2019 & INTEGER \\
     \hline continentExp & TEXT \\
     \hline
\end{tabular}

\subsection{EX2}
Functional dependencies: \\
dateRep $\rightarrow$ day \\
dateRep $\rightarrow$ month \\
dateRep $\rightarrow$ year \\
countriesAndTerritories $\rightarrow$ geoId \\
countriesAndTerritories $\rightarrow$ countryterritoryCode \\
countriesAndTerritories $\rightarrow$ popData2019 \\
countriesAndTerritories $\rightarrow$ continentExp \\
geoId $\rightarrow$ countriesAndTerritories \\
day, month, year $\rightarrow$ dateRep \\
dateRep, countriesAndTerritories $\rightarrow$ cases \\
dateRep, countriesAndTerritories $\rightarrow$ deaths \par
Two "countriesAndTerritories" have null for data both in countryterritoryCode as well as popData2019: Wallis-and-Futuna and Cases-on-an-international-conveyance-Japan. This is the only reason they do not predict other data \par
The combination on deaths and cases also can not predict any data.

\subsection{EX3}
Candidate keys: \\
dateRep, countriesAndTerritories $\rightarrow$ Predicts All \\
dateRep, geoId $\rightarrow$ Predicts All \\
day, month, year, countriesAndTerritories $\rightarrow$ Predicts All \\
day, month, year, geoId $\rightarrow$ Predicts All \\

\subsection{EX4}
The primary key I will be using is (dateRep, countriesAndTerritories), as this is the easiest to use

\section{Normalisation}

\subsection{EX5}
Partial Key Dependencies: \\
dateRep $\rightarrow$ day, month, year \\
day, month, year $\rightarrow$ dateRep \\
countriesAndTerritories $\rightarrow$ geoId, countryterritoryCode, continentExp, popData2019 \\
geoId $\rightarrow$ countryterritoryCode, continentExp, popData2019, countriesAndTerritories \\

\subsection{EX6}
\begin{tabular}{|c|c|}
     \hline
     \multicolumn{2}{|c|}{DateTable} \\
     \hline dateRep(key) & NUMERIC \\
     \hline day & INTEGER \\
     \hline month & INTEGER \\
     \hline year & INTEGER \\
     \hline
\end{tabular}
\begin{tabular}{|c|c|}
     \hline
     \multicolumn{2}{|c|}{CountryDataTable} \\
     \hline countriesAndTerritories(key) & TEXT \\
     \hline geoId & TEXT \\
     \hline countryterritoryCode & TEXT \\
     \hline popData2019 & INTEGER \\
     \hline continentExp & TEXT \\
     \hline
\end{tabular}
\\
\begin{tabular}{|c|c|}
     \hline
     \multicolumn{2}{|c|}{CaseAndDeathTable} \\
     \hline dateRep(key) & NUMERIC \\
     \hline cases & INTEGER \\
     \hline deaths & INTEGER \\
     \hline countriesAndTerritories(key) & TEXT \\
     \hline
\end{tabular} \par
I found all partial keys, and then broke down the original table so that no partial keys remained in any of the new tables

\subsection{EX7}
There are no transitive dependencies in my new tables

\subsection{EX8}
In each table, the attributes are solely dependent on the key/keys. \par
The only exception is the "CountryDataTable", where the attributes are dependent not only on "countriesAndTerritories" but also "geoId". However, while "geoId" is dependent on "countriesAndTerritories", the same is true the other way around.

\subsection{EX9}
"DateTable" and "CaseAndDeathTable" are in Boyce-Codd Normal Form, however "CountryDataTable" is not, as "geoId", which is a non-prime attribute can determine "countriesAndTerritories".

\section{Modelling}
\subsection{EX10}
After downloading the "dataset.csv" file, I ran the following lines in the terminal:
\begin{algorithm}
sqlite3 coronavirus.db \\
.mode csv \\
.import dataset.csv dataset \\
.once dataset.sql
\end{algorithm} \par
The first line creates the new database named "coronavirus.db". The second line tells SQLite that the dataset will be in a csv file format. The third line imports the dataset into a table named "dataset". Finally, the last line exports the data into a file named "dataset.sql".

\subsection{EX11}
In this exercise, I created three tables according to the specifications in EX6. The format I followed in the "ex11.sql" file for each table was:
\begin{algorithm}
CREATE TABLE TableName \\
( \par
    attribute1 TYPE, \par
    attribute2 TYPE, \par
    attribute3 TYPE, \par
    attribute4 TYPE, \\
);
\end{algorithm} \par
Afterwards I executed the following commands in the terminal to store the new tables in the "dataset2.sql" file:
\begin{algorithm}
sqlite3 coronavirus.db $<$ ex11.sql \\
sqlite3 coronavirus.db \\
.output dataset2.sql \\
.dump DateTable CountryDataTable CaseAndDeathTable
\end{algorithm} \par

\subsection{EX12}
In this exercise I inserted the appropriate data into each of the normalised tables. To insert the data, I followed the format below for each table:
\begin{algorithm}
INSERT INTO TableName(attribute-1, attribute-2, attribute-3, attribute-4) \\
SELECT attribute-1, attribute-2, attribute-3, attribute-4 \\
FROM dataset GROUP BY attribute-1;
\end{algorithm} \par
Afterwards I executed the following commands in the terminal to store the new tables in the "dataset3.sql" file:
\begin{algorithm}
sqlite3 coronavirus.db $<$ ex12.sql \\
sqlite3 coronavirus.db \\
.output dataset3.sql \\
.dump DateTable CountryDataTable CaseAndDeathTable
\end{algorithm} \newpage

\subsection{EX13}
To test my database and SQL code, I deleted every file except for "ex11.sql" "ex12.sql" and "dataset.csv". Then, I created a new database, and ran my code again on it, to get the appropriate output. 

\section{Querying}
\subsection{EX14}
In this exercise I just selected the sum of the cases and deaths columns:
\begin{algorithm}
SELECT SUM(cases), SUM(deaths) \\
FROM CaseAndDeathTable;
\end{algorithm}

\subsection{EX15}
In this exercise, I selected the "dateRep" and "cases" columns. Then to make sure I only get UK data, I made sure to get countriesAndTerritories='United-Kingdom' with the where command. Finally, I used a command called "strftime", which allows sorting a column, based on a date.
\begin{algorithm}
SELECT dateRep, cases \\
FROM dataset \\
WHERE countriesAndTerritories='United-Kingdom' \\
ORDER BY strftime("\%d/\%m/\%Y",dateRep) ASC;
\end{algorithm} \par

\subsection{EX16}
In this exercise, I selected the "dateRep", "continentExp" columns as well as the sum of "cases" and "deaths". First I grouped the output by continent and date, after which I order them based on the same criteria.
\begin{algorithm}
SELECT continentExp, dateRep, sum(cases), sum(deaths) \\
FROM dataset \\
GROUP BY continentExp, substr(dateRep,7,4)$||$substr(dateRep,4,2)$||$substr(dateRep,1,2); \\
ORDER BY continentExp, substr(dateRep,7,4)$||$substr(dateRep,4,2)$||$substr(dateRep,1,2); 
\end{algorithm} \newpage

\subsection{EX17}
In this exercise, I selected the "countriesAndTerritories" as well as calculating the number of cases and deaths as a percentage of the population, by working with the "cases" and "deaths" columns in the way described below. The printf command merely returns a formatted string with the calculated values. I chose to calculate the percentages with a 2 decimal accuracy.
\begin{algorithm}
SELECT countriesAndTerritories, printf("\%.2f",100.0*(sum(cases)*1.0/popData2019*1.0)), printf("\%.2f", 100.0*(sum(deaths)*1.0/popData2019*1.0)) \\
FROM dataset \\
GROUP BY countriesAndTerritories \\
ORDER BY countriesAndTerritories
\end{algorithm} \par

\subsection{EX18}
In this exercise, I selected the "countriesAndTerritories" as well as calculating the percentage of deaths per cases, by working with the "cases" and "deaths" columns in the way described below. I once again worked with a 2 decimal accuracy, and limited the output to the first 10 countries, ordered by death per case percentage.
\begin{algorithm}
SELECT countriesAndTerritories, printf("\%.2f",100.0*(sum(deaths)*1.0/sum(cases)*1.0)) \\
FROM dataset \\
GROUP BY countriesAndTerritories \\
ORDER BY 2 DESC \\
LIMIT 10
\end{algorithm} \par

\subsection{EX19}
In this exercise, I selected the "countriesAndTerritories" as well as calculating the cumulative deaths and cases, by working with the "cases" and "deaths" columns in the way described below. To do this I needed to order the data by date, before I did any operations, so that the cumulative sum would come out correctly. After all was calculated, I just needed to select the data regarding the UK, and order it once more.
\begin{algorithm} 
SELECT dateRep, SUM(cases) OVER (ROWS UNBOUNDED PRECEDING), SUM(deaths) OVER (ROWS UNBOUNDED PRECEDING) \\
FROM (SELECT * FROM dataset ORDER BY substr(dateRep,7,4)$||$substr(dateRep,4,2)$||$substr(dateRep,1,2) ASC) \\
WHERE countriesAndTerritories='United-Kingdom' \\
ORDER BY substr(dateRep,7,4)$||$substr(dateRep,4,2)$||$substr(dateRep,1,2);;
\end{algorithm} \par

\section{Extension}
\subsection{EX20}

\end{document}
